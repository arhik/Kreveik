\documentclass[a4paper,12t]{article}  
\usepackage{amsmath}
\usepackage{amssymb}
%\usepackage[utf8]{inputenc}
\usepackage[turkish]{babel}
\usepackage{mathrsfs}
\usepackage{graphicx}
\usepackage{lscape}
\usepackage{xkeyval}
\usepackage{lmodern}
\usepackage[T1]{fontenc}
\usepackage{subfigure}
\usepackage{gensymb}
\usepackage{bbm}
\usepackage{float}
\usepackage{bpchem}
\usepackage{natbib}
\usepackage{rotating}
\citestyle{nature}
\usepackage{color}
\usepackage{listings}
%\usepackage{savetrees}
\usepackage{hyperref}

\lstset{ %
language=Python,                % choose the language of the code
basicstyle=\small,       % the size of the fonts that are used for the code
numbers=left,                   % where to put the line-numbers
numberstyle=\small,      % the size of the fonts that are used for the line-numbers
stepnumber=1,                   % the step between two line-numbers. If it is 1 each line will be numbered
numbersep=5pt,                  % how far the line-numbers are from the code
backgroundcolor=\color{white},  % choose the background color. You must add \usepackage{color}
showspaces=false,               % show spaces adding particular underscores
showstringspaces=false,         % underline spaces within strings
showtabs=false,                 % show tabs within strings adding particular underscores
frame=false,                   % adds a frame around the code
tabsize=2,              % sets default tabsize to 2 spaces
captionpos=b,                   % sets the caption-position to bottom
breaklines=true,        % sets automatic line breaking
breakatwhitespace=false,    % sets if automatic breaks should only happen at whitespace
escapeinside={\%}{)}          % if you want to add a comment within your code
}

%XeLateX

\usepackage{fontspec}
\usepackage{xunicode}
\usepackage{xltxtra}
\newfontfeature{Microtype}{protrusion=default;expansion=default;}
%\usepackage[utf8]{inputenc}
%\setromanfont{Gentium}
%\setromanfont{Lido STF}
\setromanfont[Ligatures={Common, Historical}]{Cardo}
\setsansfont{Colaborate-Regular}
\setmonofont{Futurist Fixed-width}

\title{Genome Paketi Kılavuzu ver. 0.05}
\author{Mehmet Ali ANIL}
\date{\today}


%\let\oldhat\hat
%\renewcommand{\vec}[1]{\mathbf{#1}}
%\renewcommand{\hat}[1]{\oldhat{\mathbf{#1}}}
%\newcommand{\citemali}[1]{\footnote{\citeauthor*{#1}, \citeyear{#1}}}

\begin{document}

%\shorthandoff{=}
\maketitle

\tableofcontents

\section{Giriş}

Genome paketi, Python dilinde yazılmış, gen regülasyon ağlarının dinamiğinin incelenmesine olanak tanıyan bir modül olarak tasarlanmıştır. Bu modül, girişleri ve çıkışı arasında bir ikili fonksiyonu gerçekleyen düğümlerde oluşan ağları yaratmak, tekil ağların dinamiğini incelemek, bu ağların bir çoğunun oluşturduğu bir ailenin genetik algoritma ile nasıl evrildiğini incelemek üzere yazılmıştır. Python yorumlayıcı interaktif modunda çalıştırılarak bir ağ ya da bir aile tanımlanabilir, anında biçimlendirilebilir, ya da işleme sokulabilir. Bu kılavuzda modüldeki sınıflar (class) ve metodlar açıklanacaktır. 

\section{Fonksiyonlar}
\subsection{generate\_random}

\emph{generate\_random} fonksiyonu, rastgele ikili fonksiyonlara sahip düğümlerden \emph{n\_nodes} adet oluşturur, bunları verilen olasılıklara göre bağlar, ve daha önceden belirlenmiş \emph{scorer} adı verilen  bir kıstas fonksiyonunu da bu ağa atar. Fonksiyon aşağıdaki gibi, \emph{n\_nodes} bir tamsayı, \emph{scorer} bir fonksiyon ve \emph{probability} bir tamsayılara sahip bir tuple olacak şekilde çağrılır. 

\label{code_generate_random}
\begin{lstlisting}
generate_random(n_nodes,scorer,probability=(0.5,0.5,0.5))
\end{lstlisting}

\emph{generate\_random} fonksiyonu \emph{network} tipinde bir obje döndürür. 

\section{Sınıflar}
\subsection{network}

\emph{network} sınıfı, bir ağın temel özelliklerini ve ağın üzerinde tanımlanacak fonksiyonların tanımını içerir. Aşağıdaki gibi tanımlanan bir ağ, komşuluk matrisi ve maskesi \emph{n\_node} $\times$ \emph{n\_node} boyutunda numpy matrisi olacak şekilde verilir. \emph{state\_vec} değişkeni önceden tanımlanmamış ise rastgele atanır.

\label{code_network}
\begin{lstlisting}
yeni_ag = network(adjacency_matrix,mask,score,state_vec=None)
\end{lstlisting}

Bir ağ böyle yaratıldığında, her bir düğümün belirli bir anda edindiği değer, bir durum vektörü oluşturur, ve bir ağın durumu bu vektör üzerinden tanımlanır.
\begin{itemize}
\item \textbf{ağ.state}, bir ağın o ana kadar edindiği tüm durum vektörlerinin bir listesini, bir matris olarak edinir.
\item \textbf{ağ.adjacency}, bir ağın komuşuk matrisinin saklandığı değişkendir.
\item \textbf{ağ.n\_nodes}, bir ağın kaç düğümlü olduğunun saklandığı değişkendir. 
\item \textbf{ağ.mask}, bir ağın maskesinin saklandığı değikendir. 
\item \textbf{ağ.equilibria}, bir ağın düğüm sayısı n olmak üzere $2^n$ olası başlangıç koşulu için ağ koşturulup denge sağlandığnda, her bir başlangıç koşulune tekabül eden yörüngenin kaç iterasyonda tekrar ettiğini tutan liste.
\item \textbf{ağ.orbits} \emph{ağ.equilibria}'da anlatılan her başlangıç koşuluna tekabül eden eğrilerin oluşturduğu matristir.
\item \textbf{ağ.score}, her bir ağa atanan bir dayanıklılık değişkenidir. 
\item \textbf{ağ.mama}, her bir ağın hangi ağdan mutasyona uğradığını gösterir.
\item \textbf{ağ.children}, her bir ağın mutasyon sonucu hangi ağları oluşturduğunu listeler.
\item \textbf{ağ.scorer}, her bir ağın dayanıklılığını ölçmek için kullanılan fonksiyondur.
\end{itemize}

\subsection{Metodlar}

Burada \emph{network} tipine sahip objelerin, sınıf tanımlarındaki metodlara değinilecektir.
\subsubsection{network.print\_id}

\emph{network.print\_id} fonksiyonu ilgili ağı tanıtan bir dizi yazıyı ekrana basar

\subsubsection{title}

\bibliographystyle{plainnat}
\bibliography{Report}

\end{document}